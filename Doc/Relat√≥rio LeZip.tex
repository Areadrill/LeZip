\documentclass[a4paper,12pt,titlepage]{article}
\usepackage[portuguese]{babel}
\usepackage[utf8]{inputenc}
\usepackage[T1]{fontenc}
\usepackage{mathtools}
\usepackage{amssymb}
\usepackage{amsmath}
\usepackage{algorithm}% http://ctan.org/pkg/algorithms
\usepackage{algpseudocode}% http://ctan.org/pkg/algorithmicx
\usepackage{url}
\usepackage{texdraw}
\usepackage{gnuplottex}
\usepackage{caption}
\usepackage{subcaption}
\usepackage[section]{placeins}
\let\biconditional\leftrightarrow
\begin{document}

\title{Relatório do Trabalho de Conceção e Análise de Algoritmos}
\date{27 de abril 2015}
\author{Francisco Veiga, up201201604@fe.up.pt
 \and João Cabral, up201304395@fe.up.pt
 \and  João Mota, up201303462@fe.up.pt\linebreak
 \and Faculdade de Engenharia da Universidade do Porto}
%\input{./title_page_1.tex}
\maketitle
\tableofcontents
\newpage
\section{Introdução}

No contexto da unidade curricular de Conceção e Análise de Algoritmos, foi solicitada a concepção de um programa de compressão de ficheiros.

\section{Problemas a abordar} 
Para o presente trabalho foi pedida a concepção de um programa de compressão, de seu nome `LeZip', que se baseia em três algoritmos:\\Algoritmo de Huffman;\\Algoritmo de Lempel-Ziv-Welch;\\Run Length Encoding (acrescentado por forma a dar mais substância ao trabalho).


\section{Casos de Utilização}
Está implementada a funcionalidade que permite ao utilizador passar uma pastao ao programa e escolher o algoritmo de compressão desejado. O programa irá prontamente gerar uma pasta semelhante com os ficheiros comprimidos. O mesmo programa poderá também executar a operação inversa, gerando uma pasta semelhante à primeira.

\newpage
\section{Formalização do problema}

\subsection{Algoritmo de Huffman}
\subsubsection*{Inputs}


\subsubsection*{Outputs}


\subsubsection*{Função Objetivo}

\subsection{Algoritmo de Lempel-Ziv-Welch}
\subsubsection*{Inputs}

\subsubsection*{Outputs}

\subsubsection*{Função Objetivo}


\subsection*{Run Length Encoding}
(...)

\linebreak
\linebreak
\linebreak
\linebreak
